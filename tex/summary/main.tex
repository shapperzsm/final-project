\chapter{Summary}
``Under what conditions will cooperation emerge in a world of egoists without
central authority?'' Robert Axelrod provides justification for this study in
the first section of his book~\cite{Axelrod1984a}. Today, many examples can be
given in which cooperation has evolved in situations where, in the short term,
it may be preferable to counteract. Due to this, a class of theorems have
emerged over the past fifty years, providing explanation for the unintuitive
phenomena.

This project consists of an empirical study into these theorems, entitled
`Folk Theorems', which are key in the repeated games theory. The aims of
the project include: an in-depth review of academic literature regarding the
theorems; the execution of a large experiment based on the
`original' folk theorem of Friedman~\cite{Friedman1971} with the Iterated
Prisoner's Dilemma; and an analysis of the effects of different tournament
characteristics on the \(p\)-threshold\footnote{The \(p\)-threshold is defined
as the probability of the tournament ending for which the
least probability of defection in Nash equilibria becomes zero.} described in the folk theorems. These
ideas are extended from a third year assignment, completed by the
author, in Game Theory.

Firstly, after an introduction to the theory of one-shot and repeated games in Chapter~\ref{ch:Introduction}, a
search of the folk theorem literature is provided in Chapter~\ref{ch:Lit_Review}. This reveals the vast
directions of research in the area for the past fifty years. Many
generalisations and refinements of the folk theorem have been analysed since the
first written papers of the 1970s. The games for which the notion has been
applied to range from complete information games to games with imperfect
private monitoring. However, in certain cases, the strategies used in the proof
of these are unstable. Also, in situations where an individual deviator cannot
be identified, a much smaller set of payoffs is achieved, yielding the
so-called `Anti-Folk Theorems'. More recently, focus has been on the application
of the folk theorem to different scenarios including: computing and quantum transportation.
Finally, it is concluded that, to the best of the author's knowledge, this is
the first study to execute an experiment of this size on the folk theorem.

Following this, a detailed description of how the experiment was set-up and
executed is given in Chapter~\ref{ch:Methods}, with justifications to the specific methods and software
chosen. After considering the benefits and drawbacks of varying file formats for
storing data, it is stated that a relational database, in particular SQLite,
would be the most appropriate. This is due to: the existence of libraries in
Python enabling easier access of the data; the general robustness of
databases; and their ability to perform out of memory operations. Due to the
pre-existing game theoretic libraries, Axelrod and Nashpy, Python was chosen as
the language of implementation and ensuring good software development principles
were followed is highlighted as a priority. The volume of data which was intended
to be collected meant that remote computing was required and this is explained here. The choice of support enumeration for the calculation of
Nash equilibria and the potential issues which may be faced due to degeneracy is
also discussed.

The main analysis of the data collected is provided in Chapter~\ref{ch:Analysis}. An initial analysis discussing the characteristics of
the strategies used and the overall summary statistics is detailed, before the
\(p\)-thresholds are explored. The tournament characteristics focused on
are: the number of opponents the Defector had, and the level of standard
Iterated Prisoner's Dilemma noise
included. However, this is concluded as a non-trivial task due to the
uncertainty of degeneracy.
Also, the inevitability of randomness within the tournaments meant that a lot
more data is required. Indeed, there are three sources of
noise impacting the tournaments. On the other hand, the graphs that
were yielded are successful in visualising the folk theorem.

Finally, Chapter~\ref{ch:Conclusions} details the
conclusions of the research prior to giving recommendations for future work.
Amongst these are suggestions on how further characteristics of the tournament
set could be studied, and the potential to predict the \(p\)-threshold via regression analysis.