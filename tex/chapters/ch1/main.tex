\chapter{Introduction}

Game Theory is the study of interactive decision making and developing
strategies through mathematics \ref{Dictionary2013}. It analyses and gives
methods for predicting the choices made by players (those making a decision),
whilst also suggesting ways to improve their 'outcome'
\ref{maschler_solan_zamir_2013}. Here, the abstract notion of utility is what
the players wish to maximise. One of the earliest
pioneers of game theory is mathematician, John von Neumann who, along with
ecnomist Oskar Morgenstern, published \textit{The Theory of Games and Economic
Behaviour} in 1944 \ref{maschler_solan_zamir_2013}. This book discusses the
theory, developed in 1928 and 1940-41, by von Neumann, regarding "games of
strategy" and its applications within the subject of economics
\ref{von2007theory}. Following this, several advancements have been made in the
area, including, most notably, John Nash's papers on the consequently named Nash
Equilibria in 1950/51 \ref{nash1950equilibrium, nash1951non}. Due to the
"context-free mathematical toolbox" \ref{maschler_solan_zamir_2013} nature of
this subject, it has been applied in many areas, from Networks
(\ref{liang2012game} and \ref{1593279}) to biology (\ref{chen2019robust} and
\ref{adeoye2012application}). In this project, the main focus is on a particular
class of theorems, within game theory, known as "Folk Theorems" with application
to the game of A Prisoner's Dilemma. These will be defined and discussed in the
subsequent sections.

\subsection{An Introduction to Games}
Consider the following scenario:

\begin{center}
    Two convicts have been accused of an illegal act. Each of these prisoners,
    separately, have to decide whether to reveal information (defect) or stay
    silent (cooperate). If they both cooperate then the convicts are given a
    short sentence whereas if they both defect then a medium sentence awaits.
    However, in the situation of one cooperation and one defection, the prisoner
    who cooperated has the consequence of a long term sentence, whilst the other
    is given a deal. \ref{Knight2017}
\end{center}

This is one of the standard games in game theory known as A Prisoner's Dilemma
which has the following normal form representation:

\kbordermatrix{
    \mbox{ } & coop & defect\\ 
    coop & (3, 3) & (0,5)\\ 
    defect & (5, 0) & (1, 1)
}
where each cell contains the ordered pair (row player's payoff, column player's
payoff). Here, the payoff values as given in \ref{axelrod1980effective} are used
(see section ..).

In general a \textit{normal form} or \textit{strategic form} game is defined by
an ordered triple $G = (N, (S_i)_{i \in N}, (u_i)_{i \in N})$, where:
\begin{itemize}
    \item $N = \{1, 2,..., n\}$ is a finite set of players;
    \item $S_i$ is the set of strategies for player $i$, for all $i$ with $S =
        \{S_1 \times S_2 \times ... \times S_n\}$ is the set of all strategies
        for all players; and
    \item $u_i : S \to \mathbb{R}$ is a payoff function which takes as input the
    set of all strategies & returns the utility gained from playing a specific strategy.
\end{itemize}
% Reference