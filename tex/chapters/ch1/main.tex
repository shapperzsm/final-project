\chapter{Introduction}

Game Theory is the study of interactive decision making and developing
strategies through mathematics \ref{Dictionary2013}. It analyses and gives
methods for predicting the choices made by players (those making a decision),
whilst also suggesting ways to improve their 'outcome'
\ref{maschler_solan_zamir_2013}. Here, the abstract notion of utility is what
the players wish to maximise (see Chapter 2 in \ref{maschler_solan_zamir_2013}
for a detailed discussion on the topic of utility theory or Section 1.3
\ref{Webb2007} for a more introductory explanation). One of the earliest
pioneers of game theory is mathematician, John von Neumann who, along with
economist Oskar Morgenstern, published \textit{The Theory of Games and Economic
Behaviour} in 1944 \ref{maschler_solan_zamir_2013}. This book discusses the
theory, developed in 1928 and 1940-41, by von Neumann, regarding "games of
strategy" and its applications within the subject of economics
\ref{von2007theory}. Following this, several advancements have been made in the
area, including, most notably, John Nash's papers on the consequently named Nash
Equilibria in 1950/51 \ref{nash1950equilibrium, nash1951non}. Due to the
"context-free mathematical toolbox" \ref{maschler_solan_zamir_2013} nature of
this subject, it has been applied to many areas, from Networks
(\ref{liang2012game} and \ref{1593279}) to biology (\ref{chen2019robust} and
\ref{adeoye2012application}). In this project, the main focus is on a particular
class of theorems, within game theory, known as "Folk Theorems" with application
to the game of A Prisoner's Dilemma. These will be defined and discussed in the
subsequent sections.

\section{An Introduction to Games}
Consider the following scenario:

\begin{center}
    Two convicts have been accused of an illegal act. Each of these prisoners,
    separately, have to decide whether to reveal information (defect) or stay
    silent (cooperate). If they both cooperate then the convicts are given a
    short sentence whereas if they both defect then a medium sentence awaits.
    However, in the situation of one cooperation and one defection, the prisoner
    who cooperated has the consequence of a long term sentence, whilst the other
    is given a deal. \ref{Knight2017}
\end{center}

This is one of the standard games in game theory known as A Prisoner's Dilemma
which has the following normal form representation:

\kbordermatrix{
    \mbox{ } & coop & defect\\ 
    coop & (3, 3) & (0,5)\\ 
    defect & (5, 0) & (1, 1)
}\label{PDMatrix}

where each cell contains the ordered pair (row player's payoff, column player's
payoff). The payoff values are as given in \ref{axelrod1980effective} and are
used throughout this project.

In general a \textit{normal form} or \textit{strategic form} game is defined by
an ordered triple $G = (N, (S_i)_{i \in N}, (u_i)_{i \in N})$, where:
\begin{itemize}
    \item $N = \{1, 2,..., n\}$ is a finite set of players;
    \item $S = S_1 \times S_2, \times ... \times S_n$ is the set of strategies
    for all players in which each vector $(S_i)_{i \in N}$ is the set of
    strategies for player i \footnote{Since the game of A Prisoner's Dilemma has
    a finite strategy set for each player $S_i = \{ \text{cooperate},
    \text{defect}\} (i \in N)$, in this project only finite strategy spaces are
    considered.}; and
    \item $u_i : S \to \mathbb{R}$ is a payoff function which associates each
    strategy vector, $\textbf{s} = (s_i)_{i \in N}$, with a utility
    \footnote{'Utility' is referred to as a player's 'payoff' throughout this
    report.} $u_i(i \in N)$.
\end{itemize}
\ref{maschler_solan_zamir_2013}

Before continuing the discussion into the key notions of game theory, it needs
to be highlighted that there is an important assumption, which is central to
most studies of game theory, entitled \textit{Common Knowledge of Rationality}.
This, more formally, is an infinite list of statements which claim:
    \begin{itemize}
        \item The players are rational;
        \item All players know that the other players are rational;
        \item All players know that the other players know that they are rational; etc.    
    \end{itemize}
Assuming Common Knowledge of Rationality allows for the prediction of rational
behaviour through a processes entitled \textit{rationalisation} \ref{Knight2019}
(see section 4.5 in \ref{maschler_solan_zamir_2013} for an alternative
explanation of this assumption). 


A strategy for player $i$, $s_{i}$, is \textit{strictly dominated} if there
exists another strategy for player $i$, say $\bar{s_{i}}$, such that for all
strategy vectors $s_{-i} \in S_{-i}$ of the other players, 
$$
    u_{i}(s_{i}, s_{-i}) < u_{i}(\bar{s_{i}}, s_{i}).
$$
In this case we say that $s_{i}$ is \textit{strictly dominated} by
$\bar{s_{i}}$. Here, $s_{-i} = \{s_{1}, s_{2}, ..., s_{i-1}, s_{i+1}, ...,
s_{n}\}$, i.e. the $i$th player's strategy has been omitted. The set, $S_{-i}$,
is defined similarly. Looking at the row player's matrix of a Prisoner's Dilemma
\ref{PDMatrix} (the first entries in the ordered tuples), it is clear that
cooperation is a strictly dominated strategy. Due to the symmetricity of the
game, this is also true for the column player. \ref{maschler_solan_zamir_2013}



So far, only the pure strategies, $S_{i}=\{\text{coop}, \text{defect}\}$, have
been discussed, thus the notion of a probability distribution over $S_{i}$ is
now introduced, giving the so-called \textit{mixed strategies}:
Let $G=(N, (S_{i})_{i \in N}, (u_{i})_{i \in N})$ be a game (with each $S_{i}$
finite), then a \textit{mixed strategy} for player $i$ is a probability
distribution over their strategy set $S_{i}$. Define:
$$
\Sigma_{i} := \{\sigma_{i} : S_{i} \to [0, 1] : \sum_{s_{i} \in S_{i}}{\sigma_{i}(s_{i})} = 1\}   
$$
to be the set of mixed strategies for player $i$. Hence, observe that the pure
strategies are specific cases of mixed strategies, with $\sigma_{i} = (1, 0)$
for cooperation and $\sigma_{i} = (0, 1)$ for defection, in the example of a
Prisoner's Dilemma.