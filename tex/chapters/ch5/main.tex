\chapter{Conclusions and Recommendations}\label{ch:Conclusions}
This project looked into the Folk Theorem for the IPD and, in particular, at the
game-ending probabilities for which defection is no longer rational. The three
aims of the study, as given in Chapter~\ref{ch:Introduction}, are restated here for
convenience:

\begin{enumerate}
    \item To provide a review of past and present literature already published in
    the field of folk theorems;
    
    \item To develop a program which executes a large experiment involving
    tournaments of the IPD with differing environments to obtain
    graphs similar to those in Figure~\ref{fig:CW_plots}; and
    
    \item To perform analyses on where the \(p\)-thresholds seem to lie and
    whether it is affected by the change in the number of players, levels of
    standard PD noise, etc.
\end{enumerate}

Therefore, in this chapter, a brief section discussing the results of each aim
is provided. Then, a few shortcomings of the project are mentioned and areas in
which the research could be expanded further are highlighted.

\section{Conclusions}\label{sec:Conclusions}
In this section, each aim is explored with regards to results found and how well they were addressed. Overall, the first two aims were
successfully achieved. A well rounded knowledge in the recent research of
folk theorems was obtained and a successful experiment carried out. However,
maximal results were not obtained for the final aim due to time limitations and
the non-triviality of degeneracy.

\subsection{Aim 1: Reviewing the Literature}\label{subsec:Aim_1_concl}
The first aim was addressed in Chapter~\ref{ch:Lit_Review}. Here, a literature
search on ``folk theorem'' yielded a vast range of results which spanned the
last fifty years. It was discussed that the true origin of the `folk theorem' is
unknown however its statement and proofs appear in written work since the 1970s.
These papers focused on infinitely repeated games with subgame perfect
equilibria. Following this, many generalisations and refinements of the folk
theorem were explored. Varying types of games, for which folk theorems have been
considered, were discussed. These included: games with complete information,
games with imperfect private monitoring and games with communication, among
others. Within these papers, it was also discovered that other equilibria
(besides the more common subgame perfect equilibria) have been considered. Some
examples are: sequential equilibria, belief-free equilibria and type-k
equilibria. Throughout, there appeared to be two main assumptions required for a
folk theorem to exist. These were full dimensionality and the number of
signals in proportion to the number of actions. Since these assumptions are restricting, a few papers have attempted to find situations in which they
can be weakened whilst still obtaining a folk theorem. In contrast, an
area of research has developed which shows the instability of equilibria and
the violation of folk theorems. The chapter is concluded by a discussion on
recent applications of the folk theorem to computing, multi-agent systems and
quantum transportation. Finally, to the best of the author's knowledge, no folk
theorem experiments, of the same size as executed in this study, have been
computed before.

\subsection{Aim 2: Software Development}\label{subsec:Aim_2_concl}
Chapter~\ref{ch:Methods} discusses the set-up and execution of the IPD experiment
referred to in Aim 2. This included justifications for the use of certain
software and methods, as well as a detailed view of the data collection
algorithm. Within this, reasons were given on why the particular attributes
of the tournaments were collected. Following this, an exploration into the
file types available for data collection was provided. After considering
the advantages and drawbacks, it was decided that a binary file, in particular a
relational database file, would be the most appropriate. The chapter then
continues with an explanation of the software implementation of
the aforementioned algorithm. It provides details on how good software
development principles were followed and on the programs utilised. For a more
detailed overview, which is not as involved as the chapter, see
Chapter~\ref{ch:Comp_Report} in the Appendix. The
main experiment was executed remotely over a few weeks and how this was
achieved is explained here. Following this, how to compute the Nash equilibria
and the notion of degeneracy were addressed. Out of the three algorithms: Support
Enumeration, Vertex Enumeration and Lemke-Howson; it was decided that Support
Enumeration would be the method used for calculating Nash equilibria, due to
its robustness. The execution timings of each algorithm were also explored but
did not yield any significant differences. To conclude, the difficulties which
could be faced due to degeneracy were briefly acknowledged and possible
solutions discussed.

\subsection{Aim 3: Analysing the Thresholds}\label{subsec:Aim_3_concl}
The final aim, regarding the analysis of \(p\)-thresholds, was addressed in
Chapter~\ref{ch:Analysis}. Firstly, an initial analysis identified the main `shapes' of the graphs obtained and detailed the summary
statistics. This included looking at characteristics of the strategies, as well
as the number of equilibria obtained from each tournament. The majority of
strategies chosen were deterministic - this is because of the ratio between
deterministic and stochastic strategies implemented in the Axelrod library. When
analysing the thresholds, the minimum, mean, median and maximum
values of the \(p\)-threshold were taken, due to varying sources of noise
obscuring a clear threshold. Here, the focus was on the effects of the number
of players and standard PD noise. However, this was
concluded as a non-trivial task due to the difficulty of identifying degenerate
games and the many sources of noise. As a
result of this, significant conclusions were unable to be drawn with the
obtained plots not revealing many trends. Thus, suggestions are
provided on how further work regarding this study could reveal more significant
information. These are restated in Section~\ref{sec:Recommendations}. In
conclusion however, the experiment was successful in providing clear
visualisations for the folk theorem and it gives insight into a wide range of
directions for future research.

\section{Limitations}\label{sec:Limitations}
In this section a few shortcomings of the study are highlighted along
with justifications.

Firstly, this project included a very large empirical study on the folk theorem.
This required the development and implementation of an algorithm.
Inevitably, to ensure the functions were
accurate, clear and gave the required information, a significant amount of time
was taken. Therefore, one of the main
drawbacks was the time constraint of completing the project. Indeed, after the
successful implementation of the data collection algorithm into Python, there
was a certain time frame before analysis could start. This was to ensure enough data had
been collected. The analysis then took a portion of the time,
especially due to the data's non-triviality mentioned previously.

As a result of this, the amount of data collected was insufficient for a
large-scale statistical analysis on extremely random data. Also, the
parameters used in the algorithm were restricted by this. Only 500
repetitions of each tournament were performed but this was only able to
`stabilise' some of the payoff matrices. The rest resulted in
threshold graphs which were less clear. Thus, higher repetitions were needed but
unrealistic for the given time-frame. Moreover, the eighteen strategies
classified as `long-run' had to be omitted but, in an ideal situation, they
would have been included to obtain more information.

Furthermore, the analysis carried out only briefly considered some of
the potential environmental effects on the \(p\)-threshold. The
`randomness' of the data meant that trends were hard to infer and key
conclusions could not be drawn. Ideally, further environmental
changes, in addition to number of players and level of standard PD noise, would
have been discussed. Another factor which could have improved the situation, but
had to be omitted due to time, was the packaging of the code. Creating a fully
working Python package, with all the required functions implemented, may have
improved the remote computing stage.

Finally, degeneracy was a major limitation due to its uncertainty. Nashpy only
identifies potential degeneracy if the equilibria are `strange'. That is, if the algorithm
detects division by zero or non-probability distributions, for example. Indeed, some
games may still be degenerate even if equilibria are identified.
This implies that discussions regarding degeneracy in
Chapter~\ref{ch:Analysis} assume that Nashpy can
detect all degeneracy, which is unrealistic.

\section{Recommendations}\label{sec:Recommendations}
Throughout this study, many interesting questions have been raised as potential
directions for future work. Hence, in this section, recommendations on how this
research could be extended, to provide more insightful results, are given.

Firstly, during analysis (Chapter~\ref{ch:Analysis}), a significant proportion
of the graphs showed no \(p\)-threshold, appearing constant at zero or one.
This indicates that the threshold must lie in the intervals (0, 001) or (0.999,
1), implying the precision level was not fine enough. Therefore, it is suggested
that these tournaments are rerun, within these intervals, using a greater
precision to obtain more information. Also, an analysis of the
strategies involved and values of standard PD noise might provide a
clearer insight into reasons for this. Moreover, re-running the whole
experiment, with a finer precision could yield
more accurate thresholds.

In order to explore the effects of stochastic players on the threshold, it is
recommended that tournament sets which contain stochastic
players are rerun without them. If the threshold is different, this could indicate that stochasticity does indeed have an effect.

It is highly recommended that a separate run of the experiment is performed,
using vertex enumeration to calculate the equilibria. This could then be used in comparison
with the support enumeration data to check the reliability of the results obtained. However, care is needed when implementing vertex
enumeration as it is less robust, (see Chapter~\ref{ch:Methods}).
Furthermore, increasing the number of tournament repetitions is suggested to
observe whether clearer \(p\)-thresholds are obtained.

Regarding multivariate data analysis of this large empirical study, it may be
insightful to perform a regression or clustering algorithm. This is motivated by
the possibility of being able to approximately predict the \(p\)-threshold of a
tournament by its characteristics: strategies, level of standard PD noise etc.
Finally, the experiment could potentially be extended to consider the different
`types' of folk theorem discussed in Chapter~\ref{ch:Lit_Review}. 