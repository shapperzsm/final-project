\chapter{Introduction and Literature Review}

Game Theory is the study of interactive decision making and developing
strategies through mathematics \ref{Dictionary2013}. It analyses and gives
methods for predicting the choices made by players (those making a decision),
whilst also suggesting ways to improve their 'outcome'
\ref{maschler_solan_zamir_2013}. Here, the abstract notion of utility is what
the players wish to maximise. One of the earliest
pioneers of game theory is mathematician, John von Neumann who, along with
ecnomist Oskar Morgenstern, published \textit{The Theory of Games and Economic
Behaviour} in 1944 \ref{maschler_solan_zamir_2013}. This book discusses the
theory, developed in 1928 and 1940-41, by von Neumann, regarding "games of
strategy" and its applications within the subject of economics
\ref{von2007theory}. Following this, several advancementsnhave been made in the
area, including, most notably, John Nash's papers on the consequently named Nash
Equilibria in 1950/51 \ref{nash1950equilibrium, nash1951non}. Due to the
"context-free mathematical toolbox" \ref{maschler_solan_zamir_2013} nature of
this subject, it has been applied in many areas, from Networks
(\ref{liang2012game} and \ref{1593279}) to biology (\ref{chen2019robust} and
\ref{adeoye2012application}). In this project, the main focus is on a particular
class of theorems, within game theory, known as "Folk Theorems" with application
to the game of A Prisoner's Dilemma. These will be defined and discussed in the
subsequent sections.

\subsection{An Introduction to Games}
Consider the following scenario:
\begin{center}
    Two convicts have been accu
\end{center}