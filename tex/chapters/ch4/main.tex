\chapter{Analyses}\label{ch:Analyses}
In this chapter, an analysis of the data collected via the methods described in
the previous chapter (Chapter~\ref{ch:Methods}) is given. Firstly, a brief
initial overview is provided, where descriptive statistics of the equilibria
obtained and the overall characteristics of the strategies used is discussed.
Following this, a critical analysis of the \(p\)-thresholds obtained is carried
out. Here, the environmental effects, on the outcome of the game, discussed
include: number of opponents and level of added noise. Note, as of writing, the database currently has 
\input{src/database_code/data/se/20_01_2020/entries-in-database.txt}
entries (rows) and a total number of 
\input{src/database_code/data/se/20_01_2020/number-of-tournaments.txt}
tournament sets. 

\section{Initial Analysis}\label{sec:Initial_Analysis}
In this section, all the data (including those games which could be degenerate)
are considered. Taking a brief look at the graphs produced for each set of
tournaments, it can be seen that the main `shapes' obtained are as seen in
Figure~\ref{fig:example_graphs}.

\begin{figure}
    \begin{subfigure}{.45\textwidth}
        \centering
        \includegraphics[width=\textwidth]{folk_thm/single_game/2/0/0.0/main.pdf}
        \caption{2-player tournament set with no added noise, one stochastic player and no degeneracy identified.}\label{subfig:clear_thresh_plot}
    \end{subfigure}
    \begin{subfigure}{.45\textwidth}
        \centering
        \includegraphics[width=\textwidth]{folk_thm/single_game/6/110/0.0/main.pdf}
        \caption{6-player tournament set with no added noise, three stochastic players and potential degeneracy was yielded from 10 of the tournaments.}\label{subfig:unclear_thresh_plot}
    \end{subfigure}

    \begin{subfigure}{.45\textwidth}
        \centering
        \includegraphics[width=\textwidth]{folk_thm/single_game/5/77/0.0/main.pdf}
        \caption{5-player tournament set with no added noise, two stochastic players and five tournaments played yielded potentially degenerate games.}\label{subfig:degenerate_plot}
    \end{subfigure}
    \begin{subfigure}{.45\textwidth}
        \centering
        \includegraphics[width=\textwidth]{folk_thm/single_game/4/60/0.0/main.pdf}
        \caption{4-player game with no added noise, one stochastic player and no degeneracy identified.}\label{subfig:constant_plot}
    \end{subfigure}
    \caption{Example graphs obtained from the experiment.}\label{fig:example_graphs}
\end{figure}

In Figure~\ref{subfig:clear_thresh_plot}, a clear p-threshold of approximately
0.28 is apparent, clearly visualising the Folk Theorem. In this tournament set
the opponent was \textit{Inverse}, a stochastic player, indicating that perhaps
the stochasticity of a player does not affect the accuracy of the threshold as
first thought. In Figure~\ref{subfig:unclear_thresh_plot}, the precise value of
the p-threshold is less clear, lying approximately in the range [0.1, 0.3]. This
could be due to the potential degeneracy identified or just the element of
randomness that appears within each tournament, possibly magnified by the number
of stochastic players. The opponents within this set
were: \textit{Feld: 1.0, 0.5, 200}; \textit{Cooperator};
\textit{EvolvedLookerUp2\_2\_2}; \textit{Tullock: 11}; and \textit{ZD-GEN-2:
0.125, 0.5, 3}. Figure~\ref{subfig:degenerate_plot} shows a potential problem
with the visualisation of the Folk Theorem when degenerate games are involved.
It becomes unclear as to what is happening in the graph,
especially where the p-threshold lies. The opponents, in this case, were:
\textit{Random: 0.5}; \textit{Grumpy: Nice, 10, -10}; \textit{Fortress3}; and
\textit{Negation}. Finally, Figure~\ref{subfig:constant_plot} gives an example
of a tournament set for which there was always a non-zero probability of
defection, regardless of the game ending probability. In this case, the
precision of game-ending probabilities chosen was not accurate enough to
identify the p-threshold. It is implied that the tournament has to have an
ending probability of almost 0 (within the interval (0, 0.001) in order for a
zero probability of defection to be rational. Here, the opponents of the
\textit{Defector} were: \textit{AntiCycler}; \textit{\$e\$}; and
\textit{Stalker: (D,)}. Similarly, it is observed that some graphs obtained
were constant at 0, again indicating that the precision of game-ending
probabilities were not fine enough to highlight the p-threshold. This indicates
that the ending probability of the tournament has to lie within the interval
(0.999, 1). That is, almost immediately, the decision to defect is no longer rational.

\scalebox{0.52}{
    \begin{table}
\centering
\caption{A table of the summary statistics produced from the data of the experiment.}
\label{tab:summary_stats}
\begin{tabular}{lrrrrrr}
\toprule
{} &  experiment\_number &  number\_of\_players &  tournament\_player\_set &  num\_of\_equilibria &  least\_prob\_of\_defection &  greatest\_prob\_of\_defection \\
\midrule
mean &      107364.481228 &           5.435509 &              97.105850 &           1.913727 &                 0.342275 &                    0.459722 \\
std  &       46880.538807 &           1.726832 &              42.619612 &           2.022014 &                 0.469061 &                    0.489564 \\
min  &           0.000000 &           2.000000 &               0.000000 &           1.000000 &                 0.000000 &                    0.000000 \\
25\%  &       72231.000000 &           4.000000 &              65.000000 &           1.000000 &                 0.000000 &                    0.000000 \\
50\%  &      114641.000000 &           6.000000 &             104.000000 &           1.000000 &                 0.000000 &                    0.000000 \\
75\%  &      147396.000000 &           7.000000 &             133.000000 &           3.000000 &                 1.000000 &                    1.000000 \\
max  &      175399.000000 &           8.000000 &             159.000000 &          39.000000 &                 1.000000 &                    1.000000 \\
\bottomrule
\end{tabular}
\end{table}

}

The summary statistics gained from running the \textit{describe} method of a
pandas database is given in Table~\ref{tab:summary_stats}. From this, it can be
seen that the number of opponents the \textit{Defector} played against ranged
from one to seven, with an average of four opponents. Also, as expected, the
mean probability of the game ending encountered was 0.5. Observe that, overall,
there were \(175,399\) distinct tournaments played with a total of 159 distinct sets of strategies. 
Looking now at the statistics for Nash equilibria, it can be seen that a total
of \(823,823\) equilibrium points were calculated in this experiment, with an
average of \(1.914 \approx 2\) equilibria per game. However, observe, at least
one game obtained 39 equilibria which will be explored into later on in this
section. Considering the probabilities of defection within these equilibria,
notice that both the greatest and the least probabilities of defection ranged
from zero to one inclusive with a \(50\)th percentile of zero. But, looking at
the average values, the least probability has a mean of 0.342 and only just
above this, the greatest probability has a mean of 0.460.

Next, further descriptive statistics are calculated for the strategies. This is
to obtain a more in-depth view on the types of strategies randomly chosen to
play and their characteristics. Executing \textit{value\_counts} method on the
column of strategy names, it is observed that the player which appeared the most
times (9 times) is \textit{ZD-GEN-2: 0.125, 0.5, 3}; followed closely by
\textit{Tideman and Chieruzzi} with 7 sets of tournaments. On the other hand 38
out of the 200 strategies playing in this experiment appeared only once.
Running the \textit{value\_counts} method again, but this time on the memory
depths of the strategies found the majority of strategies to have an infinite
memory depth. On the other hand, strategies having no memory or a depth equal to
one were also significant. Considering the stochasticity of players alongside
how many appearances each strategy made yielded the following chart in
Figure~\ref{fig:stochastic_chart}.

\begin{figure}
    \centering
    \includegraphics[width=\textwidth]{folk_thm/initial_analysis/strategy_appearances.pdf}
    \caption{A plot to show the ratio of stochastic to deterministic strategies randomly chosen throughout the experiment.}\label{fig:stochastic_chart}
\end{figure}

It is clear that there is a clear bias towards deterministic strategies
in this experiment. However, this is to be expected as running the following
code:
\begin{minted}[frame=lines, framesep=2mm, fontsize=\scriptsize, bgcolor=Cornsilk]{python}
    
    len(axl.filtered_strategies(filterset={"stochastic": True})), 
    len(axl.filtered_strategies(filterset={"stochastic": False}))

    (86, 156)
\end{minted}
it can be seen that over half the strategies coded into the Axelrod library are
classed as deterministic. Looking at Figure~\ref{fig:stochastic_chart} again,
observe, the majority of deterministic strategies were executed either once or
three times. On the other hand, a large proportion of the stochastic strategies
were played twice.

Further, the number of Nash equilibria obtained for each game was analysed and
their distributions with respect to the number of opponents of the
\textit{Defector} plotted. Executing the \textit{value\_counts} method on the
`num\_of\_equilibria' column gave the conclusion that the majority of games
(131773) yielded one Nash equilibria with 28793 games obtaining 3 equilibria.
Also, the maximum number of equilibria yielded, 39, was for one game, which,
doing a search on the database, was found to be a six player game with
noise=0.1. The opponents of the \textit{Defector} were: \textit{Inverse
Punisher}; \textit{Prober}; \textit{PSO Gambler 2\_2\_2 Noise 05};
\textit{Handshake}; and \textit{More Tideman and Chieruzzi}. This game was
expected to be degenerate however, when taking a closer look at this entry in
the database, no degeneracy was identified. This could be worth looking at in
further investigation of the work. 

Figure~\ref{fig:NE_violinplot} shows the distributions of the number of Nash
equilibria as per the number of players. This visualisation turns out to not be
extremely revealing - possibly as an effect of degeneracy - however some
insights can be found here. Observe, all of the distributions
observed on this plot have a clear modal value of one, that is, irrespective of
the number of players, the majority of games yielded only one equilibria.
Moreover, there also seems to be an increase in density around 3 equilibria
which becomes more prominent as the number of players increases. As can
be seen from the plot, the variance in the number of equilibria increases with
the number of players, apart from when there were 6 players (5 opponents of the
defector), where the spread is maximum. This could be due to the 39 equilibria
gained for one game as discussed in the paragraph above. Looking now at the mean
of the distributions, observe that these are also slightly increasing as the
number of players increase. 

\begin{figure}      
    \centering
    \includegraphics[width=\textwidth]{folk_thm/initial_analysis/player_num_of_equilibria_violinplot.pdf}
    \caption{A violinplot showing the distribution of the number of equilibria obtained for varying number of players.}\label{fig:NE_violinplot}
\end{figure}



\section{Analysis of the p-Threshold}\label{sec:Analysis_of_the_p-Threshold}

Firstly, for clarity, here is a restatement of the definition of a
\textit{\(p\)-threshold}: The probability of the tournament ending for which the
least probability of defection in Nash equilibria is non-zero.

In order to analyse the \(p\)-thresholds of the tournaments, a csv file was created~\footnote{Please see Appendix~\ref{} %TODO
for the code used to obtain these files.} containing the
minimum, mean, median and maximum probabilities for each set of tournaments.
This was in order to gain as much information as possible from tournaments which
gave graphs such as Figure~\ref{subfig:unclear_thresh_plot} as described above. 
Within this file, other than the varying thresholds, the
information about the number of players, tournament strategy and level of
noise was retained. Moreover, it contained a column which identified whether any
of the strategy sets lead to possible degenerate games.

Now, an exploration into the overall \(p\)-thresholds is given.

\begin{figure}
    \begin{subfigure}{.45\textwidth}
        \centering
        \includegraphics[width=\textwidth]{folk_thm/main_analysis/min_p_threshold_hist.pdf}
        \caption{A plot to show the minimum \(p\)-thresholds.}\label{subfig:min_p_thresh}
    \end{subfigure}
    \begin{subfigure}{.45\textwidth}
        \centering
        \includegraphics[width=\textwidth]{folk_thm/main_analysis/max_p_threshold_hist.pdf}
        \caption{A plot to show the maximum \(p\)-thresholds.}\label{subfig:max_p_thresh}
    \end{subfigure}
    \caption{Plots to show the \(p\)-thresholds for all 1001 sets of tournaments.}\label{fig:min_max_p_thresh}
\end{figure}

Observe, in Figure~\ref{subfig:min_p_thresh}, the majority of minimum thresholds
were less than 0.4, with a clear modal value of 0. That is, in a significant
proportion of the tournaments, there was no probability of the game ending for
which the probability of defection was zero. Now considering
Figure~\ref{subfig:max_p_thresh}, it can be seen that the modal value for the
maximum threshold is 1.
Also, in comparison with the minimum threshold, there is a larger spread in the
density. Yet, there is still a peak around zero as with the minimum threshold
data. 

\begin{figure}
    \begin{subfigure}{.45\textwidth}
        \centering
        \includegraphics[width=\textwidth]{folk_thm/main_analysis/mean_p_threshold_hist.pdf}
        \caption{A plot to show the mean \(p\)-thresholds.}\label{subfig:mean_p_thresh}
    \end{subfigure}
    \begin{subfigure}{.45\textwidth}
        \centering
        \includegraphics[width=\textwidth]{folk_thm/main_analysis/median_p_threshold_hist.pdf}
        \caption{A plot to show the median \(p\)-thresholds.}\label{subfig:median_p_thresh}
    \end{subfigure}
    \caption{Plots to show the \(p\)-thresholds for all 1001 sets of tournaments.}\label{fig:mean_median_p_thresh}
\end{figure}


Looking at the mean and median threshold data,
Figure~\ref{fig:mean_median_p_thresh}, observe that the distributions obtained
are similar with, again, a modal value of zero, indicating there is always a
chance of defection irrespective of the game ending probability. However, in
these histograms there is also a clear peak around a threshold of 0.5. This
suggests that some of the tournaments may have several p-thresholds, perhaps due
to stochasticity, degeneracy or just inevitable randomness that appears in the 
tournaments. Indeed, obtaining the minimum and maximum
p-thresholds for those tournaments where the mean threshold was within the range
\([0.5, 0.6]\), it can be seen that, from Figure~\ref{fig:p_mean_middle_plot}
for a significant proportion, their minimum
threshold was around zero and their maximum around one. Moreover, looking closer
at three of the tournaments which satisfied the above, there is a clear
randomness within the corresponding graphs, Figure~\ref{fig:p_mean_middle_specific}.


\begin{figure}
    \centering
    \includegraphics[width=\textwidth]{folk_thm/main_analysis/p_mean_middle_data_plot.pdf}
    \caption{A plot to show the minimum and maximum \(p\)-thresholds for those tournaments which had an mean threshold within the range \([0.5, 0.6]\).}\label{fig:p_mean_middle_plot}
\end{figure}


\begin{figure}
    \begin{subfigure}{0.3\textwidth}
        \centering
        \includegraphics[width=\textwidth]{folk_thm/single_game/6/105/0.2/main.pdf}
        \caption{6-player tournament set with an additional noise level of 0.2, two stochastic players and no degeneracy was identified.}
    \end{subfigure}
    \begin{subfigure}{0.3\textwidth}
        \centering
        \includegraphics[width=\textwidth]{folk_thm/single_game/3/35/0.1/main.pdf}
        \caption{3-player tournament set with an additional 0.1 of noise, one stochastic player and no degeneracy was identified.}
    \end{subfigure}
    \begin{subfigure}{0.3\textwidth}
        \centering
        \includegraphics[width=\textwidth]{folk_thm/single_game/3/33/0.5/main.pdf}
        \caption{3-player tournament set with an additional noise level of 0.5, one stochastic player and only one tournament yielded potential degeneracy, with a game-ending prob of 0.978838383838384.}
    \end{subfigure}
    \caption{Example plots of the tournaments where the mean \(p\)-threshold was within the range \([0.5, 0.6]\).}\label{fig:mean_middle_specific}
\end{figure}

The plots contained in Figure~\ref{fig:mean_middle_specific} were sampled
randomly using the random library in Python. What is interesting here is that
they all contain varying amounts of additional noise and further analysis into
this would be beneficial in seeing if this is one of the main causes for the
inaccuracy of the thresholds. For example, looking at how many of the
tournaments with this property had an added noise level.

Before continuing onto the general overview of the thresholds for those
tournaments which led to definite non-degenerate games, a brief point is made
regarding those tournaments which, for all probabilities of the game ending, had
no positive probability of defection. Indeed, out of the 1754 total tournaments
753 of them had the above property of no apparent threshold (that is, lie within
the unobserved interval of (0.999, 1)). That is, the plots obtained
for these tournaments were (approximately). Further investigation into these
tournaments, as well as those in which the probability of
defection was always positive, is highly recommended. 

Regarding degeneracy, out of all 1754 tournaments, 372 were highlighted as
potentially leading to degenerate games.
These are omitted from the following plots in order to focus solely on
non-degenerate games.

\begin{figure}
    \begin{subfigure}{.45\textwidth}
        \centering
        \includegraphics[width=\textwidth]{folk_thm/main_analysis/non-degen_min_p_threshold_hist.pdf}
        \caption{A plot to show the minimum \(p\)-thresholds.}\label{subfig:non_degen_min_p_thresh}
    \end{subfigure}
    \begin{subfigure}{.45\textwidth}
        \centering
        \includegraphics[width=\textwidth]{folk_thm/main_analysis/non-degen_max_p_threshold_hist.pdf}
        \caption{A plot to show the maximum \(p\)-thresholds.}\label{subfig:non_degen_max_p_thresh}
    \end{subfigure}
    \caption{Plots to show the \(p\)-thresholds for all tournaments which led to non-degenerate games.}\label{fig:non_degen_min_max_p_thresh}
\end{figure}


\begin{figure}
    \begin{subfigure}{.45\textwidth}
        \centering
        \includegraphics[width=\textwidth]{folk_thm/main_analysis/non-degen_mean_p_threshold_hist.pdf}
        \caption{A plot to show the mean \(p\)-thresholds.}\label{subfig:non_degen_mean_p_thresh}
    \end{subfigure}
    \begin{subfigure}{.45\textwidth}
        \centering
        \includegraphics[width=\textwidth]{folk_thm/main_analysis/non-degen_median_p_threshold_hist.pdf}
        \caption{A plot to show the median \(p\)-thresholds.}\label{subfig:non_degen_median_p_thresh}
    \end{subfigure}
    \caption{Plots to show the \(p\)-thresholds for all tournaments which led to non-degenerate games.}\label{fig:non_degen_mean_median_p_thresh}
\end{figure}

Comparing Figures~\ref{subfig:non_degen_min_p_thresh},~\ref{subfig:non_degen_max_p_thresh},~\ref{subfig:non_degen_mean_p_thresh}
and~\ref{subfig:non_degen_median_p_thresh} with 
Figures~\ref{subfig:min_p_thresh},~\ref{subfig:max_p_thresh},~\ref{subfig:mean_p_thresh}
and~\ref{subfig:median_p_thresh}, respectively, it can be seen that, in general,
there is no significant change in the distributions of the thresholds. But,
there is a more prominent peak in Figure~\ref{subfig:mean_p_thresh} around 0.3
than in the corresponding non-degenerate plot of
Figure~\ref{subfig:non_degen_mean_p_thresh}. Further work regarding the effects
of degeneracy is advised as the above discussion seems to indicate that
degeneracy does not seem to have as much of an affect on the thresholds as initially expected.


\subsection{Effects of the Number of Players}\label{subsec:Effects_of_the_number_of_Players}
In this section, the \(p\)-thresholds will be analysed with respect to the
number of opponents the \textit{Defector} played against. Note, in this section,
only non-degenerate tournaments will be considered.


\begin{figure}
    \centering
    \begin{subfigure}{0.45\textwidth}
        \centering
        \includegraphics[width=\textwidth]{folk_thm/main_analysis/min_p_threshold_player_violinplot.pdf}
        \caption{Minimum p-threshold violinplot.}\label{subfig:min_thresh_player_violinplot}
    \end{subfigure}
    \begin{subfigure}{0.45\textwidth}
        \centering
        \includegraphics[width=\textwidth]{folk_thm/main_analysis/max_p_threshold_player_violinplot.pdf}
        \caption{Maximum p-threshold violinplot.}\label{subfig:max_thresh_player_violinplot}
    \end{subfigure}

    \begin{subfigure}{0.45\textwidth}
        \centering
        \includegraphics[width=\textwidth]{folk_thm/main_analysis/mean_p_threshold_player_violinplot.pdf}
        \caption{Mean p-threshold violinplot.}\label{subfig:mean_thresh_player_violinplot}        
    \end{subfigure}
    \begin{subfigure}{0.45\textwidth}
        \centering
        \includegraphics[width=\textwidth]{folk_thm/main_analysis/median_p_threshold_player_violinplot.pdf}
        \caption{Median p-threshold violinplot.}\label{subfig:median_thresh_player_violinplot}
    \end{subfigure}
    \caption{Violinplots of the thresholds for each number of opponents.}\label{fig:player_mean_thresh_violinplot}
\end{figure}

From Figure~\ref{subfig:mean_thresh_player_violinplot}, it can be seen that the
distributions of the minimum \(p\)-thresholds with respect to the number of
players all have a modal value around 0. However, apart from the 7-player
tournaments, the spread of the distributions decrease, along with the mean
values. Considering the maximum thresholds,
Figure~\ref{subfig:max_p_threshold_violinplot}, the distributions become bimodal
with mode values of around 0 and 1. But, as the number of players increases the
modal value at zero becomes less prominent with the 8-player tournament
distribution not having a mode around 0. The variance of the distributions are
similar and, apart from 4-player tournaments, the means increase with the number
of players. Looking now to the mean and median violinplots,
Figures~\ref{subfig:mean_thresh_player_violinplot}
and~\ref{subfig:median_thresh_player_violinplot} respectively, there is no
significant difference between the two plots other than the spread is a little
larger in the median thresholds. Within the mean plot,
Figure~\ref{subfig:mean_thresh_player_violinplot}, it can be seen that the
distributions also start off bimodal, at 0 and approximately 0.5. But again, the
modal value at zero becomes less distinct with the 8-player tournament
distribution being unimodal. The modal value at around 0.5 is a consequence of
the reason stated in the previous section. Moreover, observe that, apart from
4-player tournaments, the variance of the distributions seem to decrease with
the size of the player set whilst the means increase from a value of
approximately 0.3 for 2-player tournaments to around 0.5 for 8-players. Looking
at Figure~\ref{fig:player_mean_thresh_violinplot} as a whole, it is implied
that the number of players has no significant effect on the value of the
p-threshold.


\subsection{Effects of Noise}\label{subsec:Effects_of_Noise}
Here, an analysis on the effects of noise on the \(p\)-threshold is provided.
The addition of noise to a tournament indicates that, with a certain
probability, the action of a particular strategy is
altered~\cite{glynatsi2020meta}. That is, an action of \textit{coop} changes to
\textit{defect} and vice versa.  


\begin{figure}
    \begin{subfigure}{0.45\textwidth}
        \centering
        \includegraphics[width=\textwidth]{folk_thm/main_analysis/min_p_threshold_noise_violinplot.pdf}
        \caption{Minimum p-threshold violinplot.}\label{subfig:min_thresh_noise_violinplot}
    \end{subfigure}
    \begin{subfigure}{0.45\textwidth}
        \centering
        \includegraphics[width=\textwidth]{folk_thm/main_analysis/max_p_threshold_noise_violinplot.pdf}
        \caption{Maximum p-threshold violinplot.}\label{subfig:max_thresh_noise_violinplot}
    \end{subfigure}

    \begin{subfigure}{0.45\textwidth}
        \centering
        \includegraphics[width=\textwidth]{folk_thm/main_analysis/median_p_threshold_noise_violinplot.pdf}
        \caption{Median p-threshold violinplot.}\label{subfig:median_thresh_noise_violinplot}
    \end{subfigure}
    \begin{subfigure}{0.45\textwidth}
        \centering
        \includegraphics[width=\textwidth]{folk_thm/main_analysis/mean_p_threshold_noise_violinplot.pdf}
        \caption{Mean p-threshold violinplot.}\label{subfig:mean_thresh_noise_violinplot}
    \end{subfigure}
    \caption{Violinplots of the thresholds for each level of noise.}\label{fig:noise_p_thresh_violinplot}
\end{figure}

Figure~\ref{subfig:min_thresh_noise_violinplot}, shows the distribution of the
minimum p-thresholds for each level of additional noise added to the
tournaments. Here, it can be noted that the distributions of noise levels at
least 0.6 have a large variance, indicating that adding a too large noise
probability is highly random and thus no conclusions can be drawn. On the
other hand, considering the noise levels less than 0.6, observe that the mean
p-thresholds decrease from around 0.25 to approximately 0 as the noise
increases. These distributions are also clearly unimodal. Looking now at
Figure~\ref{subfig:max_thresh_noise_violinplot}, it can be seen that the
majority of distributions have a much larger spread here varying over the full
spectrum of game-ending probabilities and similar observations can be made from
Figures~\ref{subfig:median_thresh_noise_violinplot}
and~\ref{subfig:mean_thresh_noise_violinplot}. 

Figure~\ref{fig:single_set_vary_noise} shows an example of the same tournament set through a variety of
differing noise levels. Indeed, here, it can clearly be seen that the amount of
noise does effect the p-threshold. However, this is to be expected since, by
definition, as the noise level increases, the \textit{Defector} will be observed
as similar to the \textit{Cooperator} when there is no additional noise.

\begin{figure}
    \begin{subfigure}{0.3\textwidth}
        \centering
        \includegraphics[width=\textwidth]{folk_thm/single_game/6/103/0.0/main.pdf}
        \caption{6-player game with no additional noise.}
    \end{subfigure}
    \begin{subfigure}{0.3\textwidth}
        \centering
        \includegraphics[width=\textwidth]{folk_thm/single_game/6/103/0.1/main.pdf}
        \caption{6-player game with an additional noise level of 0.1.}
    \end{subfigure}
    \begin{subfigure}{0.3\textwidth}
        \centering
        \includegraphics[width=\textwidth]{folk_thm/single_game/6/103/0.2/main.pdf}
        \caption{6-player game with noise level 0.2.}
    \end{subfigure}

    \begin{subfigure}{0.2\textwidth}
        \centering
        \includegraphics[width=\textwidth]{folk_thm/single_game/6/103/0.3/main.pdf}
        \caption{6-player game with noise level 0.3.}
    \end{subfigure}
    \begin{subfigure}{0.2\textwidth}
        \centering
        \includegraphics[width=\textwidth]{folk_thm/single_game/6/103/0.4/main.pdf}
        \caption{6-player game with an additional noise level of 0.4.}
    \end{subfigure}
    \begin{subfigure}{0.2\textwidth}
        \centering
        \includegraphics[width=\textwidth]{folk_thm/single_game/6/103/0.5/main.pdf}
        \caption{6-player game with noise level 0.5.}
    \end{subfigure}
    \begin{subfigure}{0.2\textwidth}
        \centering
        \includegraphics[width=\textwidth]{folk_thm/single_game/6/103/0.6/main.pdf}
        \caption{6-player game with an additional noise level of 0.6. Note, the remaining noise levels of 0.7, 0.8, 0.9 and 1.0 all yielded the same constant 0 graph as represented here.}
    \end{subfigure}
    \caption{Observation of one 6-player tournament set through the varying
    levels of additional noise. There was one stochastic player and 13 out of
    the 1100 tournaments played yielded potential degenerate games. The opponents here were: \textit{Getzler};
    \textit{Punisher}; \textit{Forgiver}; \textit{GrudgerAlternator}; and \textit{GraaskampKatzen}.}\label{fig:single_set_vary_noise}
\end{figure}

Therefore, on the whole, there
are not many significant conclusions that can be made here. This implies that
the addition of noise to an already random tournament obscures any possible
visions of the threshold, especially as the magnitude increases.


\section{Conclusions and Further Work}\label{sec:Conclusions_and_Further_Work}
In this section, the beginnings of an analysis into the p-thresholds was
discussed. The effects of the number of players and level of additional noise on
these thresholds were the main focus with a brief discussion regarding the
degeneracy of games also given. This turned out to be a non-trivial task due to
the proportion of tournament sets which yielded potential degeneracy at certain
game ending probabilities and the inevitability of randomness within the
tournaments. However a few points of interest were highlighted and these are
summarised here. 

Firstly, the potential degeneracy of games yielded by the tournaments, which was
highlighted as a potential factor, at first glance appears to not create as much
of an issue as originally thought. The histograms of the p-thresholds, when the
tournaments including potential degeneracy were omitted, did not have any
significant changes when compared to the original histograms of all tournaments.
However, further exploration is advised here. Regarding the number of players in
a tournament, it was initially hypothesised that this would be a key factor in
the variability of the p-threshold. However, on obtaining the distributions of
the thresholds for tournament sets of size one
to eight, it was implied that the number of players does not have a significant
impact on the value of the threshold. Finally, the effect of additional noise on
the p-threshold was analysed. Here, it was observed that, as expected, the level
of additional noise did affect the p-threshold however there was no significant
trends appearing out of the randomness.

As stated above, this is only the very start of an analysis into the
p-thresholds described by the `original' Folk Theorem and the effects of the
varying environmental factors which appear in tournaments of the IPD. Thus many
questions regarding this are still to be researched and a few recommendations
regarding further work are now given. Firstly, it was observed that there were a
significant proportion of tournaments for which the graph remained constant at
zero or one. That is, the p-threshold was not identified using the precision of
game-ending probabilities chosen and therefore must lie within the intervals (0,
0.001) or (0.999, 1), respectively. Hence, these tournament sets could be rerun
with a much finer precision within the appropriate intervals to highlight
exactly what is happening here. Also, an analysis into the characteristics of
the strategies involved and the additional noise levels included in the
tournaments could provide a clearer insight into potential reasons for this.
Moreover, with regards to analysing the characteristics of players, it is
suggested that those tournament sets in which stochastic players were included
could be removed and the tournaments rerun. This could help in revealing whether
the stochasticity of the player has any effect on the threshold, as the author
has hypothesised. 

To check the reliability of the data collected, a second
experiment is recommended using a different algorithm for calculating the Nash
equilibria, for example vertex enumeration. This could be used in comparison
with the data already collected to identify whether the algorithms are producing
the same Nash equilibria or, more importantly, whether they identified the same
games as being degenerate. Furthermore, it is suggested that this experiment be
executed with a larger number of tournaments repeats (greater than 500) to
observe whether this `smooths' the payoff matrices with greater success to
enable for a clearer visualisation of the p-thresholds. Finally, some
multivariate data analysis of the results, for example regression, could provide
some more insights into this topic.

Overall, this chapter has been successful in visualising the Folk Theorem using
the data collection setup as explained in Chapter~\ref{} and using the plots
obtained as in Figure~\ref{subfig:clear_thresh_plot}.