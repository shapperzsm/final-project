\chapter{Literature Review}

% brief intro to chapter
The folk theorems are a class of results which generally state that for repeated
games, any feasible and individually rational payoff vector can be achieved as a
subgame perfect equilibrium if the players are patient enough~\cite{Li2019}. The
origin of these theorems is unknown however written proof and
research involving these ideas first appeared in the 1970s~\cite{Friedman1971,
Aumann1994, Rubinstein1979}. Since then, many generalisations and refinements of
the ideas have been explored, for differing games, including:
discounted games~\cite{Li2019, Fudenberg1986, Friedman1971}, sequential
games~\cite{} and stochastic games~\cite{}, to
name a few. Due to the identification of further equilibria (as compared with
the stage game), which are key in predicting future behaviour, these theorems
have been commented on as `fundamental' in the theory of non-cooperative
games~\cite{}. On the other hand, the majority of the strategies used in the
proofs assume the identification of individual deviators~\cite{Masso1989}, which may not
be realistic in certain situations. Hence, an area of research on the so-called
`anti-folk theorems' was introduced~\cite{Carmona2006, Masso1989, Peski2012}. Folk theorems appear to still be an
active area of research today~\cite{Ikeda2020, Parras2020, Wang2020} with many differing applications.
Therefore, in this chapter, a review of the literature on this topic is provided
with papers ranging from the `original' ideas in the 1970s to applications of
the theorems in 2020. 


\section{First Papers}
Which papers are cited as early work regarding the folk theorem ideas appears to
be disputed. However reading through the varying studies there are five papers
which are generally referenced as being original key
papers:~\cite{Friedman1971},~\cite{Aumann1994},~\cite{Fudenberg1986},~\cite{Benoit_1985}
and~\cite{Rubinstein1979}. Thus, in this section, each of these papers are
discussed in turn. 

In~\cite{Friedman1971}, the aim is to present a non-cooperative equilibrium
concept as 



% topics along the lines of:

% - stochastic games
% - finite ending games
% - discounted games
% - games with communication
% - public / private information games
% - recent research
% - applications of folk theorems
% - anti-folk theorems
% - 
% - 
% - 
% - 
% - 
% - 

% short summary of findings