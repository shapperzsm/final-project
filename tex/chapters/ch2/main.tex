\chapter{Literature Review}

(Friedman, 1971)~\cite{friedman1971non} is believed to be the first paper to
formally state and prove the Folk Theorem~%\cite{}
The objective of this paper was to prove the existence of a non-cooperative
equilibrium for infinitely repeated games which ``has some features resembling
the Nash cooperative solution''. Friedman also presents the idea of `temptation'
which is analogous to `threat' but makes more intuitive sense in the
non-cooperative paradigm. These notions are complemented with their application
to Oligopoly.

Firstly, the paper provides an introduction to single games, as well as giving a
proof for Nash's Theorem. Friedman also states five assumptions used, regarding
the strategy space and payoff functions, the majority of which are claimed to be
``reasonable and clear''. Then a brief section on the main definitions of
infinitely repeated games is given along with a statement highlighting that
there exist ``a large number'' of non cooperative equilibria in infinitely
repeated games. Friedman uses this to emphasise that his contribution to the
area looks at a specific group of equilibria.

Following this, another four assumptions are stated regarding the single game
and discount parameter, however, these are omitted later with a few alterations.
Friedman then defines the ``Cournot Strategy'' which is the repeated play of the
non cooperative equilibrium for the single game. This equilibrium was previously
described as being equivalent to the ``Cournot Solution''. In this section, the
focus is narrowed to consider a class of strategies with the property that the
payoff obtained is higher (for the single game) than that obtained from the
Cournot Solution. Friedman goes on to explain that a strategy,
\(\sigma_{i}^{'}\) for the \(i\)th player, defined as
follows:
\begin{equation}
    s_{i1} = s_{i}^{'},
    s_{it} = s_{j}^{'} if s_{j\tau} = s_{j}^{'} j \ne i, \tau = 1, \ldots, t-1, t=2, 3, \ldots,
    s_{it} = s_{i}^{c} otherwise.
\end{equation}
where \(s_{i}^{'}\) is a strategy whose payoff dominates the Cournot solution,
\(s_{i}^{c}\), is a non cooperative equilibrium if the discounted payoff of the
infinitely repeated game when playing  \(s_{i}^{'}\) is greater than the
discounted payoff obtained when playing \(s_{i}^{c}\) each turn after playing a
strategy which obtains maximum payoff in the first round. In addition, it is
stated that if the strategy \(s_{i}^{'}\) is contained within a subset of the
previously mentioned class such that it gives a Pareto optimal payoff vector 