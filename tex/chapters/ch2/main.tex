\chapter{Literature Review}\label{ch:Lit_Review}

The folk theorems are a class of results which generally state that for repeated
games, any feasible and individually rational payoff vector can be achieved as a
subgame perfect equilibrium if the players are patient enough~\cite{Li2019}. The
origin of these theorems is unknown, however written proof and
research involving these ideas first appeared
in~\cite{aumann1976long,Friedman1971,Rubinstein1979}\footnote{The
paper~\cite{aumann1976long} referenced here was published in 1994. However, as
implied in~\cite{Abreu1994}, this is a more recent version of a paper written in 1976.} in the 1970s. Since then,
many generalisations and refinements of
the ideas have been explored, for different games, including:
games with private monitoring~\cite{Hoerner2006,Matsushima2004, Peski2012}, sequential
games~\cite{Bhaskar1998,Gossner1996,Wen2002} and games of complete
information~\cite{Abreu1994,Benoit_1985,Bernergard2019}, to
name a few. Due to the identification of further equilibria (as compared with
the stage game), which are key in predicting future behaviour, these theorems
have been commented on as `fundamental' in the theory of non-cooperative
games by~\cite{Hoerner2006, Li2015}. On the other hand, the majority of the strategies used in the
proofs assume the identification of individual deviators~\cite{Masso1989} which may not
be realistic in certain situations. Hence, an area of research on the so-called
`anti-folk theorems' was introduced~\cite{Masso1989,Peski2012,Yoon2001}. Folk theorems appear to still be an
active area of research today~\cite{Ikeda2020, Parras2020, Wang2020} with many differing applications.
Therefore, in this chapter, a review of the literature on this topic is provided
with papers ranging from the `original' ideas in the 1970s to applications of
the theorems in 2020. 


\section{First Papers}\label{sec:First_Papers}
According to~\cite{Abreu1994}, the earliest work on Folk-type theorems is~\cite{Friedman1971}. In~\cite{Friedman1971}, infinitely repeated games with
discounting are considered. In particular, focus is on a class of strategies,
now known as \emph{grim trigger}. These are used to prove that, for any feasible
and individually rational payoff vector, there exists a discount parameter such
that a subgame perfect equilibrium with payoffs equal to that vector exist. This
is first shown under the constraints of: identical stage games, constant
discount parameter and unique Nash equilibrium that is not Pareto optimal. This
is in addition to those made on the strategy and payoff spaces. However, a more
general result is then given which removes these restrictive conditions.
The application of oligopoly is used throughout~\cite{Friedman1971}. Moreover,
he introduces the notion of `temptation'. He motivates this through the
explanation that `threat' is no longer ``credible'' since players are unable to
communicate in non-cooperative games. On the other hand, `temptation' is said to be analogous to `threat'.

In contrast to this,~\cite{aumann1976long} presents folk theorems for infinite
games without discounting, assuming payoffs take a ``limiting average form''.
This choice is justified through the statement that, as the discount rate
approaches zero, the limit of the discounted sum behaves similarly to the
limiting average payoff
form. However, trigger strategies are still required in the proof and, for
simplicity, two player games are assumed. Two folk theorems are
proven in this paper with the latter one discussing subgame perfect equilibria
similar to~\cite{Friedman1971}, whilst the former is a more generic version of
the theorem. In~\cite{aumann1976long} it is discussed that, though the generic version of
the theorem does exist, the subgame perfect equilibrium points allow for more
`believable' behaviour. They conclude by considering an example in which payoffs
are discounted. A similar approach is taken in~\cite{Rubinstein1979}, with the
statement of a folk theorem for infinite games with no discounting and the
existence of subgame perfect equilibria. The only difference being the use of an
``overtaking criterion'' instead of a ``limit of the mean
criterion'' as in~\cite{aumann1976long}.


\section{Games with Complete Information}\label{sec:Games_with_Complete_Information}
Folk theorems of games assuming complete information are studied
in~\cite{Abreu1994, Bernergard2019}. That is, all players have common knowledge
of payoff functions and strategies. Paper~\cite{Abreu1994} focuses on the
necessary and sufficient conditions required for a folk theorem proof. They
state that feasibility and individual rationality of the payoff vector are
necessary, and also approximately sufficient, conditions for a payoff to be in
the equilibrium. This is followed by a discussion on the full dimensionality
constraint first introduced by~\cite{Fudenberg1986} and often used in proving
folk theorems. It is highlighted in~\cite{Abreu1994} that the equality between
the dimension of the convex hull of feasible payoffs and the number
of players is a sufficient condition. This provides motivation for the main result of the paper; the
introduction of a `non equivalent utilities' condition, which is proved to be
sufficient and almost necessary for a folk theorem. The condition, as indicated
here, is weaker than the aforementioned full dimensionality condition. According
to~\cite{Abreu1994}, it only requires ``no pair of players have equivalent utility functions''.

Similarly, a folk theorem for complete information
games with subgame perfect equilibria is proved in~\cite{Bernergard2019}. However, instead of exponential
discounting as in~\cite{Abreu1994, Fudenberg1986}, he assumes discounting is
present-biased. That is, the discount function implies a player is more willing
to alter an event in the future than altering a current event. The folk theorem
is proved in cases of where the player is time consistent (prefers to maximise
their initial preferences) and time inconsistent (prefers to maximise current
preferences). In~\cite{Bernergard2019}, the notion of `patience' is taken to be the sum of all discount
factors. Also considering time consistent and inconsistent
players,~\cite{Li2019} discusses folk theorems with respect to time-dependent
discounting. In contrast
to~\cite{Bernergard2019}, the long-term characterisation of `patience' is taken
as the discount factors at all stages uniformly converge to one. Motivation for
the study of time-dependent discounting is explained by empirical studies which
seem to suggest that a player's time is unstationary, rather than stationary
which is assumed in most discounted game models.


\section{Games with (Imperfect) Private Monitoring}\label{sec:Games_with_(Imperfect)_Private_Monitoring}
In the late 1990s attention turned to games with imperfect monitoring. That is,
a player's actions can no longer be observed accurately, instead public or
private signals are detected~\cite{Durlauf2016}. According to~\cite{Matsushima2004}, games with (imperfect) public
monitoring were the first to be considered. More recent
studies in this area include~\cite{Chassang2011,Kandori2006}.
Repeated games with public monitoring are stated
in~\cite{Kandori2006} to give a
suitable model for studying long term relationships. They say intuition
suggests that cooperation becomes easier to sustain as action observability
improves. However,~\cite{Kandori2006} continues to show that this intuition is
false, by considering a repeated public monitored game which satisfies ``the
limit perfect public equilibrium payoff set can achieve full efficiency
asymptotically as public information becomes less sensitive to hidden actions''.
They give an example which violates the sufficient condition given
in~\cite{Fudenberg_1994} yet a folk theorem can still be obtained. This paper mainly
focuses on
the PD but, similarly to~\cite{Abreu1994}, they state an example
which satisfies the folk theorem without the full dimensionality condition. 

The notion of a robust equilibrium to incomplete
information is considered in~\cite{Chassang2011}. That is, if the equilibrium
yielded is near the original equilibrium for all perturbed games consisting of a
small independent and identically distributed information shock. Conversely to~\cite{Friedman1971},
they discuss the implication of grim trigger strategies not being robust when
the aim is to sustain cooperation. A folk theorem is proved in robust equilibria
for games of public monitoring, however~\cite{Chassang2011} highlights that a
much stronger condition is needed in comparison to the full dimensionality
requirement in~\cite{Fudenberg_1994}.

As of 2004, according to~\cite{Matsushima2004}, repeated games with private
monitoring was a relatively new area of study. In this paper, conditional
independence of signals
is assumed in order to show a folk theorem for the IPD\@. Applications to
duopolies are then described. Another study to consider a folk theorem for the
IPD is~\cite{Ely2002}. Using a similar definition of robustness
to~\cite{Chassang2011},~\cite{Ely2002} proves that for a discounted PD with private monitoring technologies, a folk theorem with robust
equilibrium strategies can be obtained. In particular, they consider
almost-perfect private monitoring and a limit folk theorem (for sequential
equilibria) follows. In a similar paper,~\cite{Ely2005} introduce the idea of
``belief-free'' equilibrium strategies; a property which implies the belief of
an opponent's history is not required when obtaining a best response. They use
these strategies in proving a folk theorem for the two-player PD
however, for general games, the set of belief-free payoffs is not large enough
to provide a folk theorem. Moreover,~\cite{Ely2005} highlight that, for a larger
number of players, the calculation of the payoff set becomes significantly
harder.

In contrast to this,~\cite{Hoerner2006} discusses a folk theorem using
strategies that, although are not belief-free, still make beliefs ``irrelevant''
at the start of each T-block of stage games. Their result is much more generic
than~\cite{Ely2005} since it applies to N-player finite games under the
assumption of full dimensionality with private, but almost-perfect, monitoring.
The T-block strategies mentioned above are modified in~\cite{Yamamoto2009,Yamamoto2012} such that they ``can support any vector in the belief-free
equilibrium payoff set''. This modification results in belief-free equilibrium
strategies. The results in~\cite{Yamamoto2009,Yamamoto2012} are generalisations
from the two-player PD in~\cite{Ely2005} to the corresponding
N-player game.


\section{Games with Communication}\label{sec:Games_with_Communication} 
Academics introduced games with communication to deal with the complications
faced with imperfect monitoring. For example, according to~\cite{Kandori2003},
games with public monitoring can obtain a folk theorem under weaker assumptions
than those given in~\cite{Fudenberg_1994}, if communication is introduced.
In this paper, communication is a message, taken from the set of possible
actions, which the players give simultaneously after choosing an action and
observing a signal. He proves a folk theorem for symmetric games
with four or more players, without the assumption of the number of signals
relevant to the number of actions.

Regarding private monitoring,~\cite{Obara2009} claims repeated games are ``very
difficult to analyse without communication''. Thus examples of papers, proving
folk theorems for these games with communication,
include~\cite{Fudenberg2008,Li2010,Obara2009}. A
`Nash threats' folk theorem is proved in~\cite{Fudenberg2008},
similar to~\cite{Friedman1971}, in the case of almost public information games,
without independent signals, for two players. In this paper, communication is
defined in the form of announcements where each set of announcements is the same
for all players. The decision to only consider two player games is justified
in~\cite{Fudenberg2008} by
highlighting that, although the results can be generalised, in certain cases it
can be seen as advantageous to have additional players. Similarly,~\cite{Obara2009} proves a `Nash threats' folk theorem for private monitoring games
with communication. He increases the number of environments where the folk
theorem is applicable through developing further the idea of ``delayed
communication'', as given in~\cite{Compte1998}. In addition,~\cite{Obara2009}
uses the assumptions of correlated private signals; and each player's deviation
from strategy is statistically identifiable from the other players' signals. A model of private monitoring and communication
within games is also considered in~\cite{Li2010}. He has the aim of increasing the number of applicable
environments for~\cite{Kandori1998} frequent
communication folk theorem. The paper states that~\cite{Kandori1998} assumed only private signals were publicised in this theorem
but if other information was useful and communication was free and legal then
players would also want to share their actions. This motivates the reasoning
behind the paper. However, assumptions of full dimensionality, and the
number of actions and signals, are still required. 

Another paper which considers communication is~\cite{Block2016}, regarding self
referential games with codes of conduct. These codes are descriptions of how the
players and opponents should play and an application to computer algorithms is
provided. Two folk theorems are proved: one assuming common knowledge of the
codes of conduct, and the other where only certain players observe certain
codes. The latter is the main result of the paper and is motivated by the fact
that, often, individuals have good knowledge of those closest to them but not
the whole community. Moreover,~\cite{Block2016} obtain the sufficient condition
that, with public communication, if every player is observed by another two
opponents, then a folk theorem is yielded.


\section{Finite Horizon Games}\label{sec:Finite_Horizon_Games}
Another area of interest is the existence of folk theorem-type results for games
of finite repetition. In~\cite{Benoit_1985}, the case of
finitely repeated games of complete information and the associated subgame
perfect equilibria is explored. Despite the existence of games which, when
repeated finitely, ``produce no non-cooperative equilibrium outcomes''; they
state there may be subgame perfect equilibria of finite-repeated games, when the
corresponding single game has multiple equilibria. Indeed, using their ``three
phase punishment'',~\cite{Benoit_1985} prove that ``any rational and feasible
payoff vector can be obtained in the limit''. This is assuming the feasible
payoff region has dimension equal to the number of players and each player has
two Nash equilibrium payoff values. In a similar manner,~\cite{ANGELOVA2011}
considers an alternative version of the PD in which an additional strategy is
included. This gives a second pure-strategy equilibrium and a folk theorem
result for the finitely repeated version of this game. 

Although not strictly a finite game,~\cite{Fujiwara-Greve2018} study a repeated
game in which players may ``strategically terminate'' it. In particular, this
involves the incorporation of a voting-step, at the start of each repetition,
where a certain number of players decide whether or not to keep interacting.
This is motivated by the increasing possibilities of ending business
partnerships due to more technology and knowledge. A general folk theorem for
any stage game (with the additional voting), which is satisfied ``for all
majority rules except the unanimous ending'' is proved by~\cite{Fujiwara-Greve2018}. Indeed, for the unanimous ending rule, they
show that the theorem may not hold but sufficient conditions are
provided for when it is satisfied.


\section{Stochastic and Sequential Games}\label{sec:Stochastic_and_Sequential_Games}
Other game types to have associated folk theorem results include
stochastic~\cite{Dutta1995} and overlapping generation~\cite{Bhaskar1998,
Gossner1996}.

It is explained in~\cite{Dutta1995} that often the standard assumption of ``a
completely unchanging environment'', within the theory of repeated games, is not
reliable in applications. This reasoning is the motivation for studying, the
more generic, stochastic games. These games may not have a pre-decided stage
game, instead a `state variable' is used to represent its environment which
alters according to``initial conditions, player's actions, and the transition
law''. The paper~\cite{Dutta1995} discusses equilibrium payoffs in the case of
very patient players, without the
need for the Markovian property. Specifically, perfect monitoring is assumed,
along with asymptotic state independence and either of payoff asymmetry or full
dimensionality. Two folk theorems are proved in~\cite{Dutta1995}: one with
unobservable mixed strategies (in which case, fully dimensionality is required)
and also, similar to~\cite{Abreu1994}, one with the slightly weaker condition of
payoff asymmetry (here, mixed strategies have to be observed).

Both~\cite{Bhaskar1998, Gossner1996} provide insight into folk theorems
associated with overlapping generation games. These are similar to the repeated
normal form games except that the players are considered to be finite.
That is, each player is involved in a certain number of stage games before they
are replaced by another, identical player. A variety
of folk theorems are proved in~\cite{Gossner1996} both with and without
discounting and / or observable mixed strategies. He shows that the full
dimensionality assumption is not required in
these games since players are assumed to not end simultaneously. On the other
hand, although~\cite{Bhaskar1998} does provide a folk theorem, his main result
is an anti-folk theorem, see Section~\ref{sec:Anti-Folk_Theorems}. He studies games
with imperfect public monitoring and states that, in such overlapping
generational games, cooperation becomes impossible in contrast to repeated
games. Furthermore,~\cite{Bhaskar1998} constructs a mixed strategy equilibrium
folk theorem. However, he goes on to show that these strategies are unstable to
perturbations, resulting in an anti-folk theorem.

A similar study by~\cite{Anderlini2008} looks at dynastic repeated games. These
differ from the overlapping generation games in two aspects: perfect observation
of the past is assumed, and the payoffs obtain no dynastic component. Under the
assumptions of full dimensionality and the existence of a payoff vector which
strictly Pareto dominates the stage game equilibrium,~\cite{Anderlini2008}
proves a folk theorem for private communication games, with greater than three
players, in sequential equilibria.

Considering now sequential games, in which players do not choose their action
simultaneously,~\cite{Wen2002} introduces a concept of effective minimax values
before proving a corresponding folk theorem. In his model, players pick actions
in groups. Thus the effective minimax value is defined to be the lowest
equilibrium payoff a player will receive, even if none of their opponents have
equivalent utilities. According to~\cite{Wen2002}, his folk theorem can be
applied to other game models as it is a ``uniform characterisation''.


\section{Anti-Folk Theorems}\label{sec:Anti-Folk_Theorems}
A common theme in the proofs of folk theorems is the use of strategies which
``identify and punish'' deviators~\cite{Masso1989}. However, as soon as the game
contains incomplete / imperfect information, deviators cannot necessarily be
identified. This yields a much smaller equilibrium set and these results are
termed ``Anti-Folk Theorems''. This description of anti-folk theorems is adapted
from~\cite{Masso1989}, who state the original term was given
in~\cite{Dubey1984,Kaneko1982}. An anti-folk theorem is proved
in~\cite{Masso1989} using the ``long-run average criterion'' instead of the
discounted criterion.

Considering similar models to~\cite{Bhaskar1998, Gossner1996}, an anti-folk theorem for a limited-observability overlapping generations
model is obtained in~\cite{Yoon2001}. They show that cooperation cannot be
sustained when new players can only observe recent history. This is in contrast
to the folk theorems obtained under
the assumption of common knowledge of all past actions. Though the results
in~\cite{Yoon2001} are
restrictive in certain cases, they justify the work by stating it is suitable
for modelling ``high turnover'' rates.

Another game model where an anti-folk theorem has resulted is a repeated game
with private monitoring~\cite{Peski2012}. In this paper the following
three assumptions are made: infinite and connected private monitoring (that is,
infinitely many connected signals); finite past; and independent and identically
distributed shocks affect the payoffs. Under these assumptions,~\cite{Peski2012}
shows the violation of the folk theorem. That is, the equilibria of the repeated
game consist only of the equilibria of the single game.


\section{Evolutionary Stability}\label{sec:Evolutionary_Stability}
Recently, there has been research into the results of the folk theorem with
respect to the evolutionary game theoretic paradigm. Indeed,~\cite{Li2015} state
that the folk theorem is often used to characterise evolutionary stable
strategies. This is since exact solutions using theoretical results from
evolutionary game theory are hard to obtain. However, they show the
assumption that the folk theorem yields all Nash equilibria is misleading. This
is achieved by defining ``type-k equilibria'' which are a refinement of the Nash
equilibria. The set of type-k equilibria is proved
in~\cite{Li2015} to be contained
within the set of repeated-game Nash equilibria using ``reactive strategies''.

In contrast to~\cite{Ely2005,Yamamoto2009,
Yamamoto2012}, who discuss folk theorems using belief-free
strategies,~\cite{Heller2017} discusses the instability in an evolutionary
sense. He shows that the belief-free equilibria are not robust to small
perturbations in games with private monitoring and, in certain cases, this is
extreme. Similar to~\cite{Li2015}, he states that Nash equilibria are used to
predict evolutionary behaviour since they are thought of as stable.
However,~\cite{Heller2017} goes on to show that only the choice of repeated stage-game Nash equilibria satisfy evolutionary stability.  


\section{Recent Applications}\label{sec:Recent_Applications}
In recent years, studies have been applying results of the folk theorem in
various areas, for example, to create algorithms. This section briefly
discusses a few of these.

The folk theorem is used in~\cite{Chowdhary2017} for the creation of a model
which aims to suppress the effects of distributed denial of service attacks.
They claim that all networks suffer from attacks to infrastructure and services. Thus,~\cite{Chowdhary2017}
use the programmability of software defined network environments to perform a
game theoretic analysis. An algorithm is created for reward and
punishment based on the Nash folk theorem. Similarly,~\cite{Wang2020} make use
of the cooperative equilibrium solution from the folk theorem
in~\cite{Friedman1971} to create an algorithm suggested to optimise a
`multi-period production planning based real-time scheduling method', for a job
shop. Also,~\cite{Wang2018} uses the result of the folk theorem in the IPD to
study the cooperation rates of varying agent strategies in a multi-agent system.

Another application of the folk theorem is an algorithm used to obtain
equilibria of a discounted repeated game~\cite{Parras2020}. A new algorithm,
entitled ``Communicate \& Agree'', is introduced in~\cite{Parras2020} to
find equilibria in incomplete information, but perfect monitoring, games. Using
the folk theorem in the algorithm enables the payoffs obtained to be potentially
higher than those achieved by repeating the Nash equilibria of the single game.
However,~\cite{Parras2020} go on to highlight that the algorithm is not always
guaranteed to find equilibria. They say it is dependent on: the discount
factor, sampling density, and whether it is a zero-sum game or not.
Finally, in a different area,~\cite{Ikeda2020} discuss the potential of using
game theoretic ideas in quantum optimal transport. In particular, he defines
the Quantum PD and explores the possibility of a quantum folk
theorem in relation to the corresponding repeated game.


\section{Conclusion}\label{sec:Conclusion}
In this section, an overview into research regarding the folk theorem has been
provided. The research history of the folk theorem spans from the 1970s until
now, with many different models being considered. Examples include: games with
complete information, games with imperfect private monitoring and finite-horizon
games. However, there have also been studies into situations where the folk
theorem does not hold, or the equilibrium strategies used in proving the
theorems are unstable and / or not robust. 